\subsection{Isosurface Extraction Algorithms}
\label{chap1:sec:prevwork}

Isosurface extraction is a popular visualization technique, being a
tool currently used in science, engineering, and applications.
This popularity makes it a natural target for this first application of
verification mechanisms in the context of visualization.
%
This same popularity has also driven a large body of work
comparing different isosurface extraction algorithms.

Previous researchers have examined topological 
issues~\cite{ning93, Lewiner:2003},
mesh quality~\cite{Dietrich:TVCG:2008,Schreiner06},
accuracy~\cite{patera04,zhou01}, and
performance~\cite{Sutton00acase}. The influence of different reconstruction schemes and 
filters in scalar visualization has also been examined~\cite{Hamish06,Pommert02}.
In this paper, we focus on techniques to verify the correctness of
algorithms and their corresponding implementations. In particular, we
provide mathematical tools that other researchers and developers can
use to increase their confidence in the correctness of their own
isosurface extraction codes.  A traditional way to test
implementations in scientific visualization is to perform a visual
inspection of the results of the Marschner-Lobb
dataset~\cite{marschnerlobb}. In the context of isosurface extraction,
researchers routinely use tools such as Metro~\cite{Cignoni:1998:MET} to
quantitatively measure the distance between a single pair of surfaces.
We argue that the methodology presented here is more effective and more
explicit at elucidating a technique's limitations. In particular, our proposal
pays closer attention to the interplay between a theoretical
convergence analysis and the experimental result of a \emph{sequence} of
approximations.

% carlos: we need to cite crj's "visualizing errors" here.. this is an
% error quantification technique.
Globus and Uselton~\cite{globus95} and more recently
Kirby and Silva~\cite{kirby-vv-08} have pointed out the need 
for verifying both visualization techniques and the corresponding
software implementations. In this paper, we provide concrete tools for
the specific case of isosurface extraction. Although this is only one
particular technique in visualization, we expect the general technique
to generalize.

% As can be seen from the discussion above, the visualization literature 
% has lagged behind in
% verifying visualization codes. Some authors argue that this fault is due to the lack
% of a specific methodology, that is, particular verification approaches must be set forth  
% in the context of visualization. The present work aims at presenting a 
% framework to verify isosurfacing tools which follows
% the well established verification methodology usually employed in the
% V\&V field. In fact, we shall show that even simple manufactured solutions
% can reveal unexpected behavior, allowing not only to better understanding
% the conduct of visualization tools but also find out errors in computational codes.

It is important to again stress that verification is a \emph{process}: even
when successfully applied to an algorithm and its implementation, 
one can only concretely claim that the implementation behaves correctly
(in the sense of analyzed predicted behavior) for all test cases to which
it has been applied.  Because the test set, both in terms of model
problems and analyzed properties, is open-ended and ever increasing, 
the verification process must continually be applied to previous and
new algorithms as new test sets become available.  This does not, however,
preclude us from formulating a basic set of metrics against which
isosurface extraction methods should be tested, as this is the starting
point of the process.  This is what we turn to in the next section.
