% Complete in less than 350 words.

In this dissertation, we advance the theory and practice of verifying visualization {algori\-thms}.
%
%%% GEOMETRY
We present techniques to assess visualization correctness  through testing of important mathematical properties. Where applicable, these techniques allow us to distinguish
 whether anomalies in visualization features can be attributed to the
  underlying physical process or to artifacts from the implementation under verification.
Such scientific scrutiny is at the heart of {\em
    verifiable visualization} -- subjecting visualization algorithms
  to the same verification process that is used in other components of
  the scientific pipeline.  The contributions of this dissertation are manifold. We derive the mathematical framework for
  the expected behavior of several 
  visualization algorithms,
  and compare them to experimentally observed results in the selected
  codes.  In the Computational Science \& Engineering community \cse, this technique is know as the Method of Manufactured Solution (MMS). We apply MMS to the verification of geometrical and topological properties of isosurface extraction algorithms, and direct volume rendering. 
%To apply MMS for the verification of these properties, we mathematically described the important properties necessary for carry the MMS test. 
We derive the convergence of geometrical properties of isosurface extraction techniques, such as function value and normals. For the verification of topological properties, we use stratified Morse theory and digital topology to
design algorithms that verify topological invariants. In the case of volume rendering algorithms, we provide the expected discretization errors for three different error sources.
The results of applying the MMS is another important contribution of this dissertation. 
%
We report unexpected behavior for almost all implementations tested. In some cases, we were able to find and fix bugs that prevented the correctness of the visualization algorithm. 
%
In particular,  we address an almost 20-year-old bug with the core disambiguation procedure of  Marching Cubes 33, one of the first algorithms intended to preserve the topology of the trilinear interpolant. 
%
Finally,  an important by-product of this work is a range of responses practitioners can expect to encounter with the visualization technique under verification. 
  