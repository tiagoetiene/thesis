\section{Conclusion}

Using a simple manufactured solution, 
we were able to reveal bugs that prevented the convergence of 
some mesh properties of two publicly available isosurfacing codes.
In particular, the by-products of the verification process, namely
a continuous refinement of mathematical analysis of the algorithm's
behavior and a numerical comparison of the results of the 
implementation against a known solution are valuable in their own right, 
and should be published together with new algorithms.
%
In the next chapter, we present a natural extension of the verification of geometrical properties of isosurfaces, namely, the verification of topological properties of isosurfaces.

%We are investigating the applicability of MMS to other
%visualization techniques such as streamline generation and volume
%rendering. In particular, MMS should clarify assumptions and
%errors intrinsic in these visualizations, a topic that has
%received recent attention\cite{Johnson:2003:NSV:942583.942610}.  More importantly, we hope the 
%examples presented here will encourage the adoption of MMS by the
%visualization community at large, 
%increasing the impact of its contributions 
%to a wider audience.

