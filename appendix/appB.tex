\chapter{Auxiliary expansions}\label{sec:aux}

In this chapter, we provide auxiliary expansions and proofs. More details on the expansions shown below can be found in the textbook by Sedgewick and Flajolet \cite{Sedgewick95}. Recall that $d = D / n$.

\section{Proof of convergence of $(1+ O(d))^n$}

Let us first expand the term $(1+C_1x)^{C_2/x}$, where $C_1 C_2 \in \mathbb{R}^+$ and $x \rightarrow 0$.
\begin{eqnarray}
(1+C_1x)^{\frac{C_2}{x}}
&= & \exp \left(\frac{C_2}{x}\log (1+C_1x) \right)\\
&= & \exp \left(\frac{C_2}{x}(C_1x + O(x^2)) \right)\\
&= & \exp ( C_1C_2 + O(x) ) \\
&=& 1 + C_1C_2 + O(x) = O(1)
\end{eqnarray}
Hence:
\begin{eqnarray}
(1+ O(d))^n = (1+ O(d))^{D/d} = O(1)
\end{eqnarray}

\section{Proof of convergence of $(1+ O(d^2))^n$}

Let us first expand the term $(1+C_1x^2)^{C_2/x}$, where $C_1,C_2 \in \mathbb{R}^+$ and $x \rightarrow 0$.
\begin{eqnarray}
(1+C_1x^2)^{\frac{C_2}{x}}
&= & \exp \left(\frac{C_2}{x}\log (1+C_1x^2) \right)\\
&= & \exp \left(\frac{C_2}{x}(C_1x^2 + O(x^4)) \right)\\
&= & \exp ( C_1C_2x + O(x^3) ) \\
&=& 1 + C_1C_2x + O(x^3)\\
&=& 1 + O(x)
\end{eqnarray}
Hence:
\begin{eqnarray}
(1+ O(d^2))^n = (1+ O(d^2))^{D/d} = 1 + O(d)
\end{eqnarray}
